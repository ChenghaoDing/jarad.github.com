\documentclass[10pt]{article}

\usepackage{verbatim,multicol,color,amsmath,ifdraft, graphicx, wrapfig,setspace}

%\usepackage[latin1]{inputenc}
%\usepackage[T1]{fontenc}
%\usepackage[dvips]{graphicx}

\title{STAT 401A Mid-term Exam \\ Friday 18 October 8:00-8:50}
\author{Instructor: Jarad Niemi}
\date{}

\newenvironment{longitem}{
\begin{itemize}
  \setlength{\itemsep}{15pt}
  \setlength{\parskip}{20pt}
  \setlength{\parsep}{20pt}
}{\end{itemize}}

\setlength{\textheight}{9in}
\setlength{\textwidth}{6.5in}
\setlength{\topmargin}{-0.125in}
\setlength{\oddsidemargin}{-.2in}
\setlength{\evensidemargin}{-.2in}
\setlength{\headsep}{0in}

\newcommand{\bigbrk}{\vspace*{2in}}
\newcommand{\smallbrk}{\vspace*{.3in}}

\ifdraft{
  \newcommand{\correct}[1]{{\color{red} #1}}
  \newcommand{\shortcorrect}[1]{{\color{red} #1}}
  \newcommand{\longcorrect}[2][\bigbrk]{{\color{red} #2}}
}{
  \newcommand{\correct}[1]{}
  \newcommand{\shortcorrect}[1]{{\phantom{33.33}}}
  \newcommand{\longcorrect}[2][\bigbrk]{#1}
}

\newcommand{\iid}{\stackrel{iid}{\sim}}
\newcommand{\Yiid}{Y_1,\ldots,Y_n\stackrel{iid}{\sim}}

\begin{document}

\maketitle


\bigskip


\textbf{INSTRUCTIONS}

\bigskip

Please check to make sure you have 5 pages with writing on the front and back (some pages are marked `intentionally left blank'). Please remove the last page, i.e. the one with SAS code on front and back. When returned to you, the peer evaluation page will also be removed.

\bigskip

On the following pages you will find short answer questions related to the topics we covered in class for a total of 50 points. Please read the directions carefully.

\bigskip

You are allowed to use a calculator and one $8\frac{1}{2}\times 11$ sheet of paper with writing on both front and back. A non-exhaustive list of items you are not allowed to use are {\bf cell phones, laptops, PDAs, and textbooks}. Cheating will not be tolerated. Anyone caught cheating will receive an automatic F on the exam. In addition the incident will be reported, and dealt with according to University's Academic Dishonesty regulations. Please refrain from talking to your peers, exchanging papers, writing utensils or other objects, or walking around the room. All of these activities can be considered cheating. {\bf If you have any questions, please raise your hand.}

\bigskip

You will be given only the time allotted for the course; no extra time will be given.

\bigskip

Some notation that should be familiar:
\begin{center}
\begin{tabular}{ll}
%$Exp(\beta)$ &: exponential distribution with mean $\beta$ \\
%$Be(\alpha,\beta)$ &: beta distribution with parameters $\alpha$ and $\beta$ \\
%$Geo(p)$ &: Geometric distribution with probability of success $p$ \\
$t_v$ &: $t$-distribution with $v$ degrees of freedom \\
%$\chi^2_v$ &: $\chi^2$-distribution with $v$ degrees of freedom \\
$F_{u,v}$ &: $F$-distribution with $u$ numerator and $v$ denominator degrees of freedom \\
%$\chi^2_{v,\alpha}$ &: the value $c$ such that $P(X>c)=\alpha$ if $X\sim \chi^2_v$ \\
$t_{v}(a)$ &: the value $c$ such that $P(X<c)=a$ if $X\sim t_v$ \\
%$z_{\alpha}$ &: the value $c$ such that $P(X>c)=\alpha$ if $X\sim N(0,1)$ \\
$\overline{y}$ &= $\frac{1}{n} \sum_{i=1}^n y_i$ \\
$s_y^2$ &= $\frac{1}{n-1} \sum_{i=1}^n (y_i-\overline{y})^2$ \\
\end{tabular}
\end{center}

\smallbrk

Good Luck!

\smallbrk

Please print your name below:

\smallbrk


Student Name: \underline{\phantom{XXXXXXXXXXXXXXXXXXXXXXXXXXXXXXXXXXXXXXXXX}}  

\newpage
\noindent \begin{Large}Short answer (10 pts total) \end{Large}

\bigskip


For each of the following tests, describe the type of data amenable to their use. (2 pts each) As an example, the One-way ANOVA F-test is used when the data are independent and normally distributed with each group having the same variance.
\begin{enumerate}
\item Wilcoxon signed rank test

\shortcorrect{Paired samples where the differences are NOT normally distributed.}\vspace*{1in}

\item Welch's t-test

\shortcorrect{Independent samples that are normally distributed with different variances}\vspace*{1in}

\item Wilcoxon rank sum test

\shortcorrect{Independent samples that are NOT normally distributed but have similar distributional shapes.}\vspace*{1in}

\item Two-sample t-test

\shortcorrect{Independent samples that are normally distributed with the same variance.}\vspace*{1in}

\item Paired t-test

\shortcorrect{Paired samples where the differences (or ratios) are normally distributed with different variances}

\end{enumerate}

\begin{comment}
\newpage
\noindent \begin{Large}``Natural'' selection (10 pts) \end{Large}

To study the effects of natural selection, scientists randomly assigned algae to high and normal levels of carbon dioxide (CO2). The algae were grown at this CO2 level for 1,000 generations. Summaries of the growth rates for alga under these two conditions are given in the table below. Perform the two-sample t-test and make a scientific conclusion about the effect of CO2 level on growth rate. 
\begin{center}
\input{table}
\end{center}
\end{comment}

\newpage
\noindent \begin{Large}Statistical inference (10 pts)\end{Large}

In a recent issue of Science, scientist conducted a study to determine the relationship between prothoracicotropic hormone (PTTH) and light avoidance in fruit fly, \emph{Drosophila melanogaster}, larva. The scientists took two vials from a set of 10 vials of genetically identical fruit flies from their lab. The first vial was assigned to be the control while the second vial was assigned to have the PTTH gene silenced. From each of these vials, the fruit flies were mated and 10 females from each vial were placed in their own new vial. For each of these new vials, light avoidance was measured on the vial. The scientists found a significant difference in light avoidance between the control and PTTH gene silenced group. (This based on a real study, page 1113 of Science 6 Sep 2013, but details have been modified.) 

\smallbrk

Identify the following (1 pt each):

\begin{itemize}
\item response variable

\shortcorrect{light avoidance}

\item explanatory variable

\shortcorrect{PTTH silencing: yes/no}

\item sample

\shortcorrect{the 2 vials are the sample from the population and the 10 new vials of each treatment where light avoidance was measured are the samples used for measurement (credit was given for either)}

\item experimental unit

\shortcorrect{the vial}

\end{itemize}

Answer the following questions (2 pts each):

\begin{itemize}
\item Are the light avoidance measurements on the new vials independent? Why or why not? 

\shortcorrect{Not independent, there is a cluster effect due to the females all originating from the same vial.}\vspace*{1in}

\item Can a causal conclusion be drawn, i.e. can the scientists say that PTTH silencing {\bf causes} a change in light avoidance? Why or why not? 

\shortcorrect{No, because the treatment (PTTH silencing) was not randomly assigned to the vial.}\vspace*{1in}

\item Can the scientists state that PTTH silencing is associated with light avoidance for all fruit flies? Why or why not? 

\shortcorrect{No, the fruit flies were not randomly selected from the population of all fruit flies.}

\end{itemize}

\newpage
\noindent \begin{Large}Thermogensis in snakes (Part I) (10 pts total) \end{Large}

Three groups of snakes were used in this study. The control group
was fasted throughout all measurements. The remaining snakes were fed varying numbers of mice to create a small meal group and a
large meal group. Infrared thermal images were taken before eating and after eating and the difference was recorded for each snake. Scientists are interested in understanding how meal size affects the temperature change.

Please answer the following questions {\bf based on the SAS code} titled ``Snakes''.

\begin{enumerate}
\item Complete the ANOVA table, i.e. fill in the XXX from the SAS ANOVA table  (7pts):

{\Large
\begin{center}
\begin{tabular}{lrrrrr}
Source & DF & Sum of squares & Mean Square & F-statistic & Pvalue \\
\hline
Model & \correct{2} & \correct{2.293} & \correct{1.146} & 28.24 & $<$0.0001 \\ \\
Error & \correct{14} & \correct{0.568} & \correct{0.041} &  & \\ \\
\hline 
Total & \correct{16} & \correct{2.861}
\end{tabular}
\end{center}
}

\longcorrect{The key is to recognize that you need MSE which you can calculate via 
\[ \frac{(n_1-1)s_1^2 + (n_2-1)s_2^2+(n_3-1)s_3^2}{(n_1-1)+(n_2-1)+(n_3-1)} = \frac{(3-1)0.1^2 + (6-1)0.93^2+(8-1)0.15^2}{(3-1)+(6-1)+(8-1)}=0.04 \]
The degrees of freedom are the number of groups minus one ($\mathrm{I}-1=2$) for the model line and number of observations minus the number of groups ($n-\mathrm{I}=14$) for the error line. Also, you can check that the math is correct by noticing that the SST is the for this table and for the regression ANOVA table. Then the remaining numbers can be calculated. The results in the table above are the actual results. Your numbers may differ due to rounding error.}

\item The ANOVA model is $Y_{ij} \stackrel{ind}{\sim} N(\mu_i,\sigma^2)$ where $Y_{ij}$ is the $j$th observation of the $i$th group.
\begin{enumerate}
\item State the null hypothesis for the pvalue in the ANOVA table. (1 pts)

\shortcorrect{$H_0:\mu_i=\mu$ for all $i$}\smallbrk

\item State the alternative hypothesis for the pvalue in the ANOVA table. (1 pts)

\shortcorrect{$H_A:\mu_i\ne \mu_k$ for some $i\ne k$}\smallbrk

\item Interpret the pvalue, i.e. what does it mean? (1 pt)

\shortcorrect{We reject the null hypothesis of equal means in all groups.}

\end{enumerate}

\end{enumerate}



\newpage
\noindent \begin{Large}Thermogensis in snakes (Part II) (10 pts total) \end{Large}

An alternative perspective for thermogensis in snakes is to consider how large the meal is as a percentage of the snake's pre-meal body mass. Scientists conducted a regression analysis using temperature change as the response and meal size as a percentage of body mass as the explanatory variable. Please answer the following questions {\bf based on the SAS code} titled ``Snakes''.

\begin{enumerate}
\item In statistical notation, write the model used in this analysis? (3 pts)\\
(Note: be sure to define any notation you introduce)

\longcorrect{The model is $Y_i\stackrel{ind}{\sim} N(\beta_0+\beta_1 X_i,\sigma^2)$ where $Y_i$ is the temperature change for snake $i$ and $X_i$ is the meal size as a percentage of body mass for snake $i$.}

\item Find the following numbers on the SAS output and provide an interpretation for them.


\begin{enumerate}
\item 0.807975 (1 pt)

\shortcorrect{This is the proportion of variation in the response described by the line.}\smallbrk

\item 0.3195592500 (1 pt)

\shortcorrect{This is the expected temperature change when the meal size is zero, i.e. no meal. }\smallbrk

\item 0.0273840185 (1 pt)

\shortcorrect{This is the expected difference in temperature change when the meal size is increased by 1.}\smallbrk

\item 0.03662784 (1 pt)

\shortcorrect{This is the estimated population variance, i.e. $\hat{\sigma}^2$.}\smallbrk


\end{enumerate}

\item Construct a 95\% confidence interval for the slope, i.e. the coefficient for mealSize. (3 pts)

\shortcorrect{We need $t_{15}(0.975) = 2.131$. The confidence interval is 
$0.027 \pm 2.131(0.003) = (0.021,0.033)$.}\smallbrk

\end{enumerate}

% \newpage
% (intentionally left blank)

\begin{comment}
\newpage
\noindent \begin{Large}Mid-term peer evaluation  (10 pts total) 
\end{Large}

\bigskip

Please assign scores (higher is better) that reflect how you really feel about the extent to which the other members of your team contributed to your learning and/or your team's performance. This will be your only opportunity to reward the members of your team who worked hard on your behalf. (Note: If you give everyone pretty much the same score you will be hurting those who did the most and helping those who did the least.)

{\bf Instructions:} In the space below please rate each of the {\bf other} members of your team. Each member's peer evaluation score will be the average of the points they receive from the other members of the team. To complete the evaluation you should: 1) List the {\bf last name} of each member of your team in the alphabetical order of their last names and, 2) assign an average of ten points to the {\bf other} members of your team (Thus, for example, you should assign a total of 50 points in a six-member team; 60 points in a seven-member team; etc.) and, 3) differentiate some in your ratings; for example, you must give at least one score of 11 or higher (maximum=15) and one score of 9 or lower.

\vspace{0.2in}

{\Large
\begin{tabular}{l@{\qquad}c@{\qquad}c@{\qquad}l@{\qquad}c@{\qquad}c@{\qquad}}
& Team Members & Score & & Team Members & Score \\
\hline
1) & & & 5) & & \\
\hline
2) & & & 6) & & \\
\hline
3) & & & 7) & & \\
\hline
4) & & & 8) & & \\
\hline
\end{tabular}
}

\vspace{0.3in}

{\bf Additional Feedback:} In the space below would you also briefly describe your reasons for your highest and lowest ratings. These comments will be aggregated and distributed to the class. Provide at least {\bf one positive critique} and {\bf one area for improvement}.

\vspace{0.2in}

{\bf Positive critique:}

\vspace{2in}

{\bf Area of improvement:} 

\newpage
(intentionally left blank)
\end{comment}

\newpage
\noindent SAS Code - Snakes

{\small
\begin{verbatim}
DATA snakes;
  INFILE 'snakes.csv' DSD FIRSTOBS=2;
  INPUT mealSize temperatureChange;
  group = 'aNone';
  IF mealSize > 0.1  THEN group = 'bSmall';
  IF mealSize > 25.1 THEN group = 'cLarge';

PROC MEANS DATA=snakes;
  CLASS group;
  VAR temperatureChange;
  RUN;

                                       The MEANS Procedure

                             Analysis Variable : temperatureChange 
 
                  N
       group    Obs     N            Mean         Std Dev         Minimum         Maximum
       ----------------------------------------------------------------------------------
       aNone      3     3       0.2000000       0.1000000       0.1000000       0.3000000
       bSmal      6     6       0.9333333       0.2804758       0.6000000       1.2000000
       cLarg      8     8       1.2250000       0.1488048       1.0000000       1.4000000
       ----------------------------------------------------------------------------------

                                        The GLM Procedure

                                    Class Level Information
                           Class         Levels    Values
                           group              3    aNone bSmal cLarg 
 
Dependent Variable: temperatureChange   

                                               Sum of
       Source                      DF         Squares     Mean Square    F Value    Pr > F
       Model                       XX      XXXXXXXXXX      XXXXXXXXXX      28.24    <.0001
       Error                       XX      XXXXXXXXXX      XXXXXXXXXX                     
       Corrected Total             XX      XXXXXXXXXX  

                                        The GLM Procedure
 
Dependent Variable: temperatureChange   

                                               Sum of
       Source                      DF         Squares     Mean Square    F Value    Pr > F
       Model                        1      2.31175884      2.31175884      63.11    <.0001
       Error                       15      0.54941763      0.03662784                     
       Corrected Total             16      2.86117647                                     

                 R-Square     Coeff Var      Root MSE    temperatureChange Mean
                 0.807975      20.33455      0.191384                  0.941176


                                                  Standard
                Parameter         Estimate           Error    t Value    Pr > |t|
                Intercept     0.3195592500      0.09097737       3.51      0.0031
                mealSize      0.0273840185      0.00344692       7.94      <.0001
\end{verbatim}
}

\begin{comment}
                                        The GLM Procedure
 
Dependent Variable: temperatureChange   

                                               Sum of
       Source                      DF         Squares     Mean Square    F Value    Pr > F

       Model                        2      2.29284314      1.14642157      28.24    <.0001

       Error                       14      0.56833333      0.04059524                     

       Corrected Total             16      2.86117647                                     


                 R-Square     Coeff Var      Root MSE    temperatureChange Mean

                 0.801364      21.40753      0.201483                  0.941176
\end{comment}

% \newpage
% (intentionally left blank)

\end{document}

