\documentclass[12pt]{article}

\usepackage[margin=2cm]{geometry}
\usepackage{hyperref,verbatim}
\hypersetup{colorlinks=true,urlcolor=black}


\begin{document}

{\Large
\begin{tabular}{@{}l}
Iowa State University \\
Department of Statistics  \\
STAT 544, Bayesian Statistics  \\
Spring 2018 \\
\end{tabular}
} % End \LARGE

\bigskip

\begin{tabular}{@{}ll@{\hspace{.2in}}ll}
Instructor: &Jarad Niemi & Office: &Snedecor 2208 \\
Email: &\href{mailto:niemi@iastate.edu}{\texttt{niemi@iastate.edu}} & Phone: & 515.294.8679 \\
Course hours: & TR 8:00--9:20 in Snedecor 3105 & Office hours: & TBD \\
TA: & Luis Damiano (\href{mailto:ldamiano@iastate.edu}{\texttt{ldamiano@iastate.edu}}) & Office hours: & TBD \\
\\
\multicolumn{4}{@{}l}{Course webpage: on Canvas} and at \url{http://jarad.me/courses/stat544} \\
\multicolumn{4}{@{}l}{Textbook:} \\
\multicolumn{4}{l}{\hspace{0.1in}
\href{http://www.stat.columbia.edu/~gelman/book/}{
{\normalsize Gelman et al (2013). Bayesian Data Analysis. CRC Press LLC. 3rd ed. }}} \\

\multicolumn{2}{@{}l}{Prerequisites:} \\
\multicolumn{4}{l}{\hspace{0.1in} previous or concurrent enrollment in STAT 543 (or Econ 672)}
\end{tabular}

\bigskip

\subsubsection*{Course description}

Specification of probability models; subjective, conjugate, and noninformative prior distributions; hierarchical models; analytical and computational techniques for obtaining posterior distributions; model checking, model selection, diagnostics; comparison of Bayesian and traditional methods. 

\subsubsection*{Course objectives}
\begin{itemize}
\item Understand the basics of a Bayesian analysis including prior, likelihood, and posterior. 
\item Perform a conjugate Bayesian analysis with Jeffreys prior.
\item Perform a computational Bayesian analysis using JAGS or Stan.
\end{itemize}

\subsubsection*{Assessment}

Homework ($\sim$ 10): 20\%, Midterm (Mar 8): 40\%, Project (due May 1): 40\%

\subsubsection*{Reading schedule}

The table below provides a reading schedule for the semester. `M' indicates the midterm and $\vert\vert$ indicates spring break. 

\vspace{0.2in} 

\begin{tabular}{|l|ccccccccc||cccccc|}
\hline
Week & 1 & 2 & 3 & 4 & 5 & 6 & 7 & 8 & 9 & 10 & 11 & 12 & 13 & 14 & 15 \\
\hline
Chapter & 1 & 2  & 3 & 4 & 5 & 6 & 7.4--7.6, 9.1 & 14 & M & 10 & 11 & 12 & 13 & 15 & 16\\
\hline
\end{tabular}

\subsubsection*{Faculty Senate Recommendations}

This course abides by the Faculty Senate Recommendations provided at \url{http://www.celt.iastate.edu/teaching/preparing-to-teach/recommended-iowa-state-university-syllabus-statements}.

\end{document}

